\subsection{Overview}
The TrackMe services are built on a client-server structure, this way the system is organized through abstraction levels.
We chose to adopt a 3-tier architecture:
\begin{itemize}
	\item \textbf{Presentation Tier} \\This layer makes the interaction possible between the user and the system. Here the user sees all the information provided by the system in a easily way to understand them.
	\item \textbf{Application Tier} \\This layer is managed almost totally by Data4Help service that is in charge of:
	\begin{itemize}
		\item store data incoming from the external;
		\item collect data information from database in order to execute Third parties’ requests;
		\item also generates data statistics on data collected;
		\item send to third parties requested data.
	\end{itemize}
	Moreover even AutomatedSOS has logic application in order to continuously monitor users’ health status.
	\item \textbf{Data Tier} \\In this layer all the sensible users’ data (location, health status) are stored into Databases and are retrieved by the application tier in order to do statistics and answer third parties’ requests.
\end{itemize}

More specifically Data4Help manage the data and core logic sections while AutomatedSOS and Track4Run manage the presentation section. Actually, a small part of application tier is also present in AutomatedSOS, this is due to the fact that the Health Monitoring process requires to be executed as fast as possible.

\subsection{Component View}

\includegraphics[scale=0.5]{Images/ComponentDiagram.png}

{\large \textbf{Component diagram description}}
\begin{enumerate}
\item [1] \textbf{ApplicationServer} 
This huge component is in charge of manage Data4Help services so store and provide, for interested companies, users’ data (such as GPS location and daily Health Status). Moreover it manages race information of Track4Run application.

	\begin{enumerate}
	\item [1.1] \textbf{WebServer:} In order to accept and supply information to who need them, this component offer a friendly web interface to simplify these operations
		
	\item [1.2] \textbf{RequestManager:} In order to support web server in its job, this component manage all the incoming request: it sorts them per urgency, it wrap the request in a smart data structure and send it to Data Collector, it unwrap the answer from Data Collector and it continuously generates data exchange if a live acquisition is active
		
	\item [1.3] \textbf{DataCollector: } This component is in charge to communicate with Database in order to retrieve information and supply them to Request Manager. To perform these operations the component should receive request from Request Manger, unwrap it, creates a query to database that cover the question and send it to database; once the database has responded it should wrap the answer and provide it to Request Manager. Moreover if a statistic on data is required it requires it to Statistic Generator.
		
	\item [1.4] \textbf{StatisticGenerator: } This component is in charge of generates statistic on data provided by data collector such as arithmetic mean, variance from average, standard deviation and median.

	\item [1.5] \textbf{UserManager: } This component is in charge to communicate with users' device in order to log (or sign up) user and retrieve data collected.

	\item [1.6] \textbf{Track4RunManager: } This component is in charge to communicate with Track4Run applications in order to allow promoter to manage races.
	
	\end{enumerate}
	
\item [2] \textbf{Database} 
This component is in charge of store physically data and organize them in a smart way according to DBMS rule, moreover it allows the access to those data.

	\begin{enumerate}
	\item [2.1]\textbf{UserDBMS: }
	This component is in charge of store all the users’ acquired data and quickly provide them to Data collector whenever it is required.
	
	\item [2.2]\textbf{Track4RunDBMS: }
	This component is in charge of store all the races' informations provided by promoters.

	
	\end{enumerate}
	
\item [3]\textbf{PartnerApp: }
This component implements how the application partner of TrackMe is basically structured, from the components in charge of retrieve raw data from device's API to who is in charge to communicate with the Main Server. This component is extended by all the partner applications that want to exploit Data4Halp service, so even by AutomatedSOS and Track4Run.

	\begin{enumerate}
	\item [3.1]\textbf{GPSAcquirer:}
	This component is developed in order to acquire GPS location from user's device at constant time period.
	
	\item [3.2]\textbf{HSAcquirer:}
	This component is developed in order to acquire Hearth rate, Blood pressure and Calories consumed from user's device at constant time period. (Obviously if device support it). 
	
	\item [3.3]\textbf{UserLogger:}
	 This component offers to client the possibility to become Data4Help user, so it is in charge to show to client the policy to accept and acquire all the credentials inserted during registration. Moreover has to communicate to Main Server the registration of a new user.
	 
	\item [3.3]\textbf{DataSender:}
	This component, exploiting GPSAcquirer and HSAcquirer features, is in charge to provide to UserManager component inside main server all the retrieved data whenever are ready to be sent.
	\end{enumerate}

\item [4]\textbf{AutomatedSOSApp: }
This component should extend all that is specified in partner application component to exploit all its features. AutomatedSOS component,also, has to use (as Data Sender) the HSAcquirer in order to check health status parameter and call the first aid whenever is required.

	\begin{enumerate}
	\item [4.1]\textbf{HealthAnalyser: }
	This component exploiting HSAcquirer features is in charge to analyse continuously health parameters, compare last acquired data with historical data in order to improve the reaction time, check data in order to prevent user's diseases and call an ambulance whenever such parameters are below a certain threshold, compiling and providing also a special report.
	
	\item [4.2]\textbf{HSAcquirer:}
	This component is developed in order to acquire Hearth rate, Blood pressure and Calories consumed from user's device at constant time period. (Obviously if device support it). 
	\end{enumerate}
	

\item [5]\textbf{Track4RunApp: }
This component should extend all that is specified in partner application component to exploit all its features. Track4Run component,also, should provide to spectator the possibilities to select and spectate to a run,also allows promoter to create and manage a race providing all the useful information.
	
		\begin{enumerate}
	\item [5.1]\textbf{RaceDisplayer: }
	 This internal component is in charge to provide a human interface to the user that allows him to specify the race that he want to spectate and then show the position of all the athletes in the race, exploiting Google Maps API.
	
	\item [5.2]\textbf{RaceManager: }
	This internal component is in charge to provide to user the possibility to promote a run inside the system, date,path,name,maximum number of partecipants and description. Moreover a promoter can invite specific users providing their fiscal code. 
	\end{enumerate}

\end{enumerate}

\subsection{Deployment View}
The following Deployment Diagram captures the topology of the system's hardware.
The SmartphoneApp and SmartWatchApp (Presentation Tier) communicate to the Application Server through RMI, while the WebBrowser communicates to the WebServer through HTTP protocol. The Application Server (Application Tier) communicates to the Database Server (Data Tier) through JDBC.

\begin{figure}[H]
\centering
\includegraphics[scale=0.35]{Images/DeploymentDiagram.png}
\caption{Deployment Diagram.}
\end{figure}

\subsection{Runtime View}
In this section several Sequence Diagrams are shown in order to point up the interaction among components and their behavior in particular scenarios.

\subsubsection{Individual Request}

\subsubsection{}
\subsubsection{}


\subsection{Component Interfaces}

\begin{enumerate}

\item[1.1] \textbf{WebServer: }
	\begin{enumerate}
		\item[1.1.1] \textbf{Web Interface:} Human interface for third parties that want to log in or retrieve information from Data4Help service. Software interface for Track4Run and for AutomatedSOS in order to provide requested information.
		\\\textit{ boolean: thirdPartyRegistration(String: Name,String Piva)
				\\ boolean: thirdPartyLogIn(String: Name)
				\\ void: makeRequests (Request[] requests)
				 }
	\end{enumerate}
	
\item[1.2] \textbf{RequestManager: }
	\begin{enumerate}
		\item[1.2.1] \textbf{AcceptInformationRequest:} Software interface that allows WebServer to submit requests to be evaluated.
		\item[1.2.1] \textbf{ProvideInformation:} Software interface that provides to web server information answers that match previous requests.
	\end{enumerate}

\item[1.3] \textbf{DataCollector: }
	\begin{enumerate}
		\item[1.3.1] \textbf{AcceptdataRequest:} Software interface offered by DataCollector that accept requests from RequestManager formatted in the proper way.
		\item[1.3.1] \textbf{ProvideInformation:} Software interface offered by DataCollector that provide to Request Manager information answer formatted in the proper way.
	\end{enumerate}
	
\item[1.4] \textbf{StatisticGenerator: }
	\begin{enumerate}
		\item[1.4.1] \textbf{ProvideStatistics:} Software interface that provides statistic values to data collector. (The Software interface to receive data from Data collector to be managed is trivial and not explicitly specified).
	\end{enumerate}
	
\item[1.5] \textbf{UserManager: }
	\begin{enumerate}
		\item[1.5.1] \textbf{AcquireData:} Software interface that allows partner application to send data acquired to Data4Help.
		\item[1.5.1] \textbf{Enroll user:} Software interface that allows partner application's users to become registered users into TrackMe's system.
	\end{enumerate}

\item[1.6] \textbf{Track4RunManager: }
	\begin{enumerate}
		\item[1.6.1] \textbf{EditRaceSettings:} Software interface that allows Track4Run application to create or modify races information.
	\end{enumerate}

\item[2.1] \textbf{UserDBMS: }
	\begin{enumerate}
		\item[2.1.1] \textbf{StoreUserData:} Software interface that allows to store users' data into the database.
		\item[2.1.2] \textbf{ProvideStoredData:} Software interface that allows to retrieve users' data already collected in the database.
	\end{enumerate}
	
\item[2.2] \textbf{Track4RunDBMS: }
	\begin{enumerate}
		\item[2.2.1] \textbf{StoreRaceData:} Software interface that allows to to store races' data with queries.
		\item[2.2.2] \textbf{ProvideStoredData:} Software interface that allows to retrieve users' data already collected in the database.
	\end{enumerate}
	
\item[3.3] \textbf{UserLogger: }
	\begin{enumerate}
		\item[3.3.1] \textbf{Enrollment:} Software interface that permits to inform Users' Manager that a new user is successfully enrolled and with which credentials.
	\end{enumerate}
	
\item[4.1] \textbf{HealthAnalyser: }
	\begin{enumerate}
		\item[4.1.1] \textbf{CallAmbulance:} Software interface that allows AutomatedSOS to call, through the internet or in the worst case through telephone call, First Aid whenever is necessary in order to provide help to user. Moreover it provides to FirstAid a special report on which parameters are critical.
	\end{enumerate}
	
\item[5.1] \textbf{RaceDisplayer: }
	\begin{enumerate}
		\item[5.1.1] \textbf{SelectRace:} Human interface that allows user to specify the race to be displayed.
		\item[5.1.1] \textbf{ShowLiveRace:}  Human interface that allows to provide to the end user a map in which all the athletes involved in the selected race are displayed.
	\end{enumerate}	

\item[5.2] \textbf{RaceDisplayer: }
	\begin{enumerate}
		\item[5.2.1] \textbf{ManageRace: } Human interface that allows user to create and manage a run providing all the useful information.
	\end{enumerate}		

\end{enumerate}




\subsection{Selected Architectural Styles and Patterns}
\subsection{Other Design Decisions}