\noindent
This section shows how the requirements can be supplied by the designed components.

\begin{enumerate}

\item [G.1] Acquire user’s position and health status.
	\begin{enumerate}
	\item [ R.1] Retrieve user credentials inserted into partner application as group attributes.
		\begin{enumerate}
		\item[•] UserLogger: Receives user's credentials from SignAccount interface and then stores them locally. After this operation it waits for the privacy policy acceptance.
		\end{enumerate}

	\item [ R.3] Allow individuals to agree the privacy policy (first part) so that they can be tracked in group mode through installed application.  
		\begin{enumerate}
		\item[•] UserLogger: Receives from SignAccount interface that the user has accepted the first part of the privacy policy. Then if the user decided to create the account, it passes all these information to the WebInterface.
		\item[•] WebServer: Receives from WebInterface the credentials and provides them to the CredentialManager through CredentialInterface.
		\item[•] CredentialManager: Checks the inserted credentials, if they are valid it stores them into the DB using DBInterface.
		\item[•] DBMS: Stores data incoming from DBInterface.
		\end{enumerate}
	
	\item [R.4] During the registration allow individuals to specify if they are also interested to be tracked in single mode (agree the second part of privacy policy) through installed application.
		\begin{enumerate}
		\item[•] UserLogger: Receives from SignAccount interface that user has accepted the second part of the privacy policy. Then if the user decides to create the account, it passes all these information to the WebInterface.
		\item[•] WebServer: Receives from WebInterface the credentials and provides them to CredentialManager through CredentialInterface.
		\item[•] CredentialManager: Checks the inserted credentials, if they are valid it stores them into the DB using DBInterface.
		\item[•] DBMS: Store data incoming from DBInterface.
		\end{enumerate}

	\item [R.2] Allow users already registered in Data4Help world to sign in with their account without providing user credentials again.
		\begin{enumerate}
		\item[•] UserLogger: This component automatically inform the WebServer whenever a user starts the application.
		\item[•] WebServer: Receives users' log requests from WebInterface and sends them to the CredentialManager.
		\item[•] CredentialManager: Checks the users' availability and communicate it in reverse order allowing the UserLogger to decide if a user is admissible or not.
		\end{enumerate}
	
	\item [R.5] For each user registered the system has to automatically retrieve and store data from partner applications with a resolution of 10 minutes; independently from the requests reached.
		\begin{enumerate}
		\item[•] DataSender: This component, exploiting the data acquired by GPSAcquirer and HSAcquirer, sends every ten minutes all the information acquired since the last message to the UserDataManager using the AcquireData interface.
		\item[•] UserDataManager: When a data update is arrived through the AcquireData interface, this component generates and sends an SQL query that allows the insertion of the new data into the DB.
		\item[•] DBMS: Stores data incoming from the DBInterface.
		\end{enumerate}	
	\end{enumerate}

\item [G.2] Provide to third parties, the user’s position and health status.
	\begin{enumerate}
	\item [R.6] Allows third parties to register to Data4Help service specifying all their credentials.
		\begin{enumerate}
		\item[•] WebServer: using the WebInterface, it provides to third parties a web page that allows them to insert all the necessary credentials; then it retrieves these information. Moreover it sends these data to the CredentialManager using CredentialInterface.
		\item[•] CredentialManager: Checks the inserted credentials  and if they are valid it stores them into the DB using DBInterface.
		\item[•] DBMS: Stores data incoming from DBInterface.
		\end{enumerate}
		
	\item [R.7] The system should allow third parties to send information requests.
		\begin{enumerate}
		\item[•] WebServer: Using the WebInterface, it provides to third parties a web page that allows them to make a request inserting all the necessary information. Then it passes the requests using the AcceptInformationRequest interface.
		\item[•] RequestManager: Receives requests from AcceptInformationRequest interface and stacks them in a queue sorted by urgency criteria.
		\end{enumerate}	
	\end{enumerate}

\item [G.2.1] Provide data on demand to non-subscribed third parties.
	\begin{enumerate}
	\item [R.8] The system has to collect all the useful data that match the request.
		\begin{enumerate}
		\item[•] RequestManager: This component periodically checks the queue in order to select the most urgent request. Once an on-demand request is picked up it wraps the request and send it to the DataCollector.
		\item[•] DataCollector: Once a data request is received, this component builds a DB query that matches what third party wants and sends it to the DB using DBInterface.
		\item[•] DBMS: When the query is arrived, this component computes it and provides data to the DataCollector using DBInterface.
		\end{enumerate}	
			
	\item [R.9 ] The system has to generate a statistic on data selected.
		\begin{enumerate}
		\item[•] DataCollector: Once data are arrived from the DB, this component asks to compute some kind of statistic on the data through the ProvideStatistic interface.
		\item[•] StatisticGenerator: This component receives data from ProvideStatistic interface, computes statistics and sends them back to the DataCollector.
		\end{enumerate}	
		
	\item [R.10] The system has to send to the third party all the raw data collected until the moment of the request.
	\item [R.11] The system has to send all the statistics already produced.
		\begin{enumerate}
		\item[•] DataCollector: Once data are arrived from the DB and statistics are generated, this component wraps them and sends this package to the RequestCollector using the ProvideCollectedData interface.
		\item[•] RequestCollector: This component receives information from DataCollector, unwraps and sends them using ProvideInfromation interface to the WebServer.
		\item[•] WebServer: Using WebInterface, it provides to third parties a web page that allows them to see statistics and download retrieved information.
		\end{enumerate}	
	\end{enumerate}

\item [G.2.2] Provide data in real-time to subscribed third parties.
	\begin{enumerate}
	\item [R.12] Allow third parties to subscribe to groups or individuals in order to receive live data.
		\begin{enumerate}
		\item[•] RequestManager: Once a real-time request is acquired this component stores all the attributes selected in order to make the subscription. Using a parallel thread it automatically adds into the priority queue the data request to supply the live acquisition.
		\item[•] DataCollector: Once a data request is received, this component builds a DB query in order to retrieve only new data that matches what the third party wants and it sends it to DB using the DBInterface.
		\item[•] DBMS: It computes the received query and provides data to the DataCollector using the DBInterface.
		\end{enumerate}	
			
	\item [R.13] Provide to subscribed third parties raw data as soon as they are available by the system.
		\begin{enumerate}
		\item[•] DataCollector: Once data are arrived from DB, this component wraps them and sends this package to the RequestCollector using ProvideCollectedData interface.
		\item[•] RequestCollector: This component receive information from DataCollector, unwraps and sends them using the ProvideInfromation interface to the WebServer.
		\item[•] WebServer: Using WebInterface, it provides to third parties a web page that allows them to download retrieved information.
		\end{enumerate}	
	\end{enumerate}
	
\item [G.3.1] Allow third parties two different ways to get users’ data.
	\begin{enumerate}
	\item [R.14] Allow third parties to insert the fiscal code of the user he wants to track.
		\begin{enumerate}
		\item[•] WebServer: In this specific case this component has to offer to third parties a web page that allows them to fill in the fiscal code that they want to track. Then the request formulated has to be forwarded to the RequestManager through the AcceptInformationRequest interface.
		\end{enumerate}	
		
	\item [R.8] The system has to collect all the useful data that match the request.
	\item [R.15] Deny third parties to receive single mode information about users that have not accepted the second part of the privacy policy.
		\begin{enumerate}
		\item[•] RequestManager: This component periodically checks the queue in order to select the most urgent request. Once a single mode request is picked up it wraps the request and sends it to the DataCollector.
		\item[•] DataCollector: Once a data request is received, this component builds a DB query that matches what third parties wants and it sends it to the DB using the DBInterface, requiring also all the information of the user involved.
		\item[•] DBMS: Compute the received query and provides data to the DataCollector using DBInterface.
		\item[•] DataCollector: Once data are retrieved this component checks, before it returns an answer to the RequestManager, if the user involved has accepted the second part of the privacy policy: if he/she did it wraps the information package and proceeds as always, otherwise it communicates to the RequestManager that the request cannot be accepted.
		\end{enumerate}	
		
	\item [R.10] The system has to send to the third party all the raw data collected until the moment of the request.
	\end{enumerate}
	
\item [G.3.2] Allow third parties to get data of a group of people.
	\begin{enumerate}
	\item [R.16] Allow third parties to insert search area and attributes in which they are interested to restrict their field of search.
		\begin{enumerate}
		\item[•] WebServer: In this specific case this component has to offer to third parties a web page that allows them to fill in all the attributes that the users to track must have. Then the request formulated has to be forwarded to the RequestManager through the AcceptInformationRequest interface.
		\end{enumerate}	
		
	\item [R.8] The system has to collect all the useful data that match the request.
	\item [R.17] Deny third parties to receive information if the provided information can hurt users' privacy, for this purpose group request under 1000 users involved are rejected.
		\begin{enumerate}
		\item[•] RequestManager: This component periodically checks the queue in order to select the most urgent request. Once a group mode request is picked up it wraps the request and sends it to the DataCollector.
		\item[•] DataCollector: Once a data request is received, this component builds a DB query that matches what the third party wants and it sends it to the DB using DBInterface, requiring also all the number of users involved in the specific query.
		\item[•] DBMS: This component computes the received query and provides data to the DataCollector using DBInterface.
		\item[•] DataCollector: Once data are finally retrieved this component checks, before it returns an answer to RequestManager, if the number of users involved are less then 1000; if so it wraps the information package and proceeds as always, otherwise it communicates to the RequestManager that the request cannot be accepted.
		\end{enumerate}	
		
	\item [R.10] The system has to send to the third party all the raw data collected until the moment of the request.
	\end{enumerate}
	
\item [G.4] Provide data in an anonymous way, to protect users' privacy.
	\begin{enumerate}
    \item [R.15] Deny third parties to receive single mode information about users that have not accepted the second part of the privacy policy.
    \item [R.17] Deny third parties to receive information if the provided information can hurt users' privacy, for this purpose group request under 1000 users involved are rejected.
    \end{enumerate}	


\item [G.5] Retrieve user's position and health status.
	\begin{enumerate}
	\item [R.18] Allow users to be tracked from AutomatedSOS filling up the registration and agreeing only to first part of privacy policy.
		\begin{enumerate}
		\item[•] UserLogger: Receives from the SignAccount interface that user has filled all the fields on the registration page and has accepted the first part of privacy policy (the only one presented) then, if the user decides to create an account, it passes all these information to the WebInterface.
		\item[•] WebServer: Receives the credentials from the WebInterface and provides them to the CredentialManager through CredentialInterface.
		\item[•] CredentialManager: Checks the credentials inserted and if they are valid it stores them into the DB using DBInterface.
		\item[•] DBMS: Stores data incoming from the DBInterface.
		\end{enumerate}
		
	\item [R.19] The application has to retrieve users' health status every 2 seconds in order to guarantee a reaction time of 5 seconds.
		\begin{enumerate}
		\item[•] HealthAnalyser: Exploiting the data acquired by GPSAcquirer and HSAcquirer, this component acquires health status parameters every 5 seconds and the GPS location every ten minutes like before.
		\end{enumerate}	
	\end{enumerate}	
	
\item [G.6] Monitor user's health parameters.
	\begin{enumerate}
	\item [R.19] The application has to retrieve users' health status every 2 seconds in order to guarantee a reaction time of 5 seconds.
		\begin{enumerate}
		\item[•] HealthAnalyser: Exploiting the data acquired by GPSAcquirer and HSAcquirer, this component acquires health status parameters every 5 seconds and the GPS location every ten minutes like before.
		\end{enumerate}	
	\item [R.20] The application sends to Data4Help service all the data retrieved in live acquisition.
		\begin{enumerate}
		\item[•] DataSender: Exploiting the data acquired by GPSAcquirer and HSAcquirer, this component sends every ten minutes all the information acquired since the last message to the UserDataManager using AcquireData interface.
		\item[•] UserDataManager: When a data update is arrived through the AcquireData interface, this component generates and sends an SQL query that allows the insertion of the new data into the DB.
		\item[•] DBMS: Stores data incoming from the DBInterface.
		\end{enumerate}	
	\item [R.21] The application gets from Data4Help service all the historical data about the user.
		\begin{enumerate}
		\item[•] HealthAnalyser: When the user requires it or when it is necessary to check the health parameters, this component sends a request to the WebServer in order to retrieve all the historical data about the user using WebInterface. From this point the sequence of action takes the standard procedure of managing requests (R.8 , R.10) until the WebServer.
		\item[•] WebServer: Once the retrieved information is arrived to the WebServer, this component sends the package to the HealthAnalyser component.
		\end{enumerate}
	\item [R.22] Allow the user to set personal threshold values.
		\begin{enumerate}
		\item[•] UserInteractionManager: This component has to show to the users a human interface that allows them to personalize the threshold parameters of their health status.
		\item[•] HealthAnalyser: Exploiting the service offered by UserInteractionManager, this component has to store (and to use them when necessary) all the settings generated by the user.
		\end{enumerate}
	\end{enumerate}

\item [G.7] Send an ambulance to user's location whenever certain parameters are below the threshold.
	\begin{enumerate}
	\item [R.23] The application has to control health status with data retrieved in local to immediately realize whether certain parameters are critical.
		\begin{enumerate}
		\item[•] HealthAnalyser: This component periodically checks the last parameters acquired and compares them to the historical ones and then decides with a special algorithm if the user needs an ambulance.
		\end{enumerate}	
	\item [R.24] The application sends an ambulance request to the nearest hospital whenever parameters are critical.
	\item [R.25] Supply to the hospital the user's location and all the useful information to provide efficient first aid.
		\begin{enumerate}
		\item[•] HealthAnalyser: When the analysis finds something bad, this component sends an ambulance request to the nearest hospital specifying the user's location and health status.
		\end{enumerate}	
	\item [R.26] In the case no answer arrives from the hospital the software must repeat another time the request until an answer in reached.
		\begin{enumerate}
		\item[•] HealthAnalyser: When an ambulance request is sent, this component actives a trigger that resend the request if an answer from the hospital is not received in time.
		\end{enumerate}	
	\item [R.27] As soon as the acknowledgement message is received a warning message is displayed on the user's smartwatch.
		\begin{enumerate}
		\item[•] HealthAnalyser: When an answer is finally arrived from the hospital this component, exploiting UserInteractionManager, displays the ETA of the ambulance.
		\end{enumerate}	
	\end{enumerate}
	
\item [G.5] Retrieve user's position and health status.
	\begin{enumerate}
	\item [R.3] Allow individuals to agree the privacy policy (first part) so that they can be tracked in group mode through installed application.  
	\item [R.4] During the registration allow individuals to specify if they are also interested to be tracked in single mode (agree the second part of privacy policy) through installed application. 
	\item [R.28] The application has to interact with Smartwatch/Smartphone APIs in order to retrieve GPS location with a resolution of 10 seconds when the user is in the run.
		\begin{enumerate}
		\item[•] DataSender: Exploiting the data acquired by GPSAcquirer and HSAcquirer, this component sends every ten seconds all the information acquired since the last message to the UserDataManager using AcquireData interface when the run is active.
		\item[•] UserDataManager: When a data update is arrived through the AcquireData interface, this component generates and sends an SQL query the allows the insertion of the new data in the DB.
		\item[•] DBMS: Stores data incoming from the DBInterface.
		\end{enumerate}	
	\end{enumerate}
	
\item [G.8] Allow user to manage a run.
	\begin{enumerate}
	\item [R.30] Allow promoters to define a path for the run by selecting the routes inside a map.
		\begin{enumerate}
		\item[•] RunManager: This component allows the promoters to specify, using ManageRun interface, the path of the race, displaying a map in which the promoter has to select the track.
		\end{enumerate}	
	\item [R.31] Allow promoters to send a participation request.
		\begin{enumerate}
		\item[•] RunManager: This component allows the promoters to specify, using ManageRun interface, the athletes that the promoter wants to invite to the run.
		\end{enumerate}	
	\item [R.32] Allow promoters to specify maximum number of athletes that can participate.
		\begin{enumerate}
		\item[•] RunManager: This component allows the promoters to specify, using ManageRun interface, the number of allowed athletes.
		\end{enumerate}	
	\item [R.29] Allow users to create a run once all the general information are inserted.
		\begin{enumerate}
		\item[•] RunManager: Once all the information above are submitted and once the name and the description of the run are also specified, this component provides all these information to the WebServer.
		\item[•] WebServer: Receives from WebInterface the information about the new race and forwards it to the Track4RunManager component.
		\item[•] Track4RunManager: Checks all the information about the new run that the user wants to create and if they are good, the component creates and sends an SQL query in order to store these information to the DBMS.
		\item[•] DBMS: Stores data incoming from the DBInterface.
		\end{enumerate}	
	\end{enumerate}	
	
\item [G.9] Allow athlete to enroll on a specific run.
	\begin{enumerate}
	\item [R.33] Allow the user to see all the runs generated (which he is invited or not).
		\begin{enumerate}
		\item[•] RunManager: When user wants to see all the runs in the system, this component sends a request to the WebServer.
		\item[•] WebServer: Receives from the WebInterface the request to see all the runs and forwards it to the RequestManager using AcceptInformationRequest interface.
		\item[•] RequestManager: This component periodically checks the queue in order to select the most urgent request. Once an on-demand request is picked up it wraps the request and sends it to the DataCollector.
		\item[•] DataCollector: Once a data request is received, this component builds a DB query that matches what the user wants and sends it to the DB using DBInterface.
		\item[•] DBMS: When the query is arrived, this component computes it and provides data to the DataCollector using DBInterface, then the information goes in the opposite direction to the RunManager.
		\item[•] RunManager: When all the runs are finally arrived, this component shows to the user all the race in which he can participate.
		\end{enumerate}	
	\item [R.35] Deny user to enroll in a run if maximum number of participants is already reached.
	\item [R.34] Allow user to enroll in a run.
		\begin{enumerate}
		\item[•] RunManager: When user wants to enroll in a run, this component retrieves through ManageRun interface which run he/she is interested in; then it provides this information to the WebServer.
		\item[•] WebServer: Receives from the WebInterface the request to enroll in a run and forwards it to the Track4RunManager using RunInterface.
		\item[•] Track4RunManager: This component has to create a query that asks to the DBMS if the interested user can participate to the specific run.
		\item[•] DBMS: This component computes the received query and provides an answer to the DataCollector using DBInterface
		\item[•] Track4RunManager: If the answer from DBMS is positive the component creates and sends another query that asks the DBMS to enroll the user, then it communicates it to WebServer; In the other case it communicates to the WebServer that the user cannot be enrolled.
		\item[•] DBMS: Stores data incoming from DBInterface.
		\item[•] WebServer: Receives from Track4RunManager the answer about the enrolling and forwards it to the RunManager through WebInterface.
		\item[•] RunManager: When the answer is arrived, this component displays to the user the result of the enrollment using ManagerRun interface.
		\end{enumerate}	
	\end{enumerate}
	
\item [G.10] \textbf{Allow spectators to watch in real time the position of every athletes in a specific run.}
	\begin{enumerate}
	\item [R.33] Allow the user to see all the runs generated (which he is invited or not).
	\item [R.36] Allow user to select a run to be viewed.
		\begin{enumerate}
		\item[•] RunDisplayer: When user wants to spectate to a run, this component has to retrieve through the ShowLiveRun interface which run he/she is interested in; then it creates and sends a live acquisition request for the specific user to the WebServer.
		\end{enumerate}	
	\item [R.37] The application requests to Data4Help the position of all the other athletes involved.
		\begin{enumerate}
		\item[•] WebServer: Receives from WebInterface the live acquisition request and forwards it ton the RequestManager using AcceptInformationRequest interface.
		\item[•] RequestManager: Once a real-time request is arrived, this component stores all the attributes selected in order to make the subscription. Using a parallel thread the component automatically add into the priority queue the data request to supply the live acquisition.
		\item[•] DataCollector: Once a data request is received this component builds a DB query in order to retrieve only new data that matches what user wants and it sends it to the DB using DBInterface.
		\item[•] DBMS: When the query is received this component computes it and provides data to the DataCollector using DBInterface.
		\end{enumerate}	
	\item [R.38] The application receives and displays the position of all the other athletes involved.
		\begin{enumerate}
		\item[•] DataCollector: Once data are arrived from DB, this component wraps them and sends this package to the RequestCollector using ProvideCollectedData interface.
		\item[•] RequestCollector: This component receive information from DataCollector, unwraps and sends them using ProvideInfromation interface to the WebServer.
		\item[•] WebServer: Provides to the RunDisplayer all the positions of the athletes involved in the race using WebInterface.
		\item[•] RunDisplayer: Using the GoogleMapsAPI interface to compute the live positions, this component offers to the user a live map that specifies all the athletes' position inside the race path.
		\end{enumerate}	
	\end{enumerate}
	
\end{enumerate} 