\subsection{Purpose} 
The following Requirements Analysis and Specification Document examines a possible solution for a specific system-to-be provided by the TrackMe company. Therefore, this document contains the description of the scenarios, the use cases that described them, and the models describing requirements and specification for the system-to-be.
\bigbreak
\noindent
Data4Help is a location-based health information service-to-be that allows third parties to monitor the location and health status of individuals. The given problem is to design and develop this service and other two services, AutomatedSOS and Track4Run, which exploit the features offered by the first one.
\bigbreak
\noindent
AutomatedSOS is a service-to-be thought to help elderly people. Constantly monitoring the health status of the subscribed customers, this service sends to the user's location an ambulance as soon as the recorded values are anomalous, for example when some health parameters are below certain thresholds.
\bigbreak
\noindent
Finally, Track4Run is a service-to-be that tracks athletes participating in a run. The service, allows organizers to define the path for the run, participants to enroll to the run and spectators to see on a map the position of all the runners during the run.
\subsection{Scope}
\subsubsection{Goals}

\begin{enumerate}
\item[•] {\Large Data4Help}
	\begin{enumerate}
		\item [G.1] Acquire user's position and health status.
		\item [G.2] Provide to third parties user's position and health status.
		\begin{enumerate}
		\item [G.2.1] Provide data on demand to non-subscribed third parties.
		\item [G.2.2] Provide data in real-time to subscribed third parties.
		\end{enumerate}
		\item [G.3] Allow third parties two different ways to get user's data.
		\begin{enumerate}
		\item [G.3.1] Allow third parties to get data of a single person.
		\item [G.3.2] Allow third parties to get data of a group of people.
		\end{enumerate}
		\item [G.4] Provide data in an anonymous way, to protect user's privacy.
	\end{enumerate}
	
\item[•] {\Large AutomatedSOS}
	\begin{enumerate}
		\item [G.5] Retrieve user's position and health status.
		\item [G.6] Monitor user's health parameters.
		\item [G.7] Send an ambulance to user's location whenever certain parameters are below the threshold.
		\end{enumerate}
	
\item[•] {\Large Track4Run}	
	\begin{enumerate}
		\item [G.5] Retrieve user's position and health status.
		\item [G.8] Allow promoters to manage a run.
		\begin{enumerate}
		\item [G.8.1] Allow promoters to define the path for the run. 
		\item [G.8.2] Allow promoters to invite athletes to the run. 
		\end{enumerate}
		\item [G.9] Allow athletes to enroll on a specific run.
		\item [G.10] Allow spectators to watch in real time the position of every athlete in a specific run.
	\end{enumerate}
\end{enumerate}

\subsubsection{World Phenomena}
... what are world phenomena???

\subsection{Definitions, Acronyms, Abbreviations}

\begin{enumerate}
\item[•] {\Large Definitions}
	\begin{enumerate}
		\item Single request: request to access to the data of some specific  individuals.
		\item Group request: request to access to anonymized data of gropus of individuals.
		\item Live/real-time acquisition/: third parties can access to data as soon they are ready, 				through service updates.
		\item On demand acquisition: third parties can access to data when they request 				them.
		\item Subscribers: third parties allowed to receive live acquisition about 						preselected	user/group.
		\item User credentials: information that an individual has to provide to become a 				registered user: name, surname, date of birth, address, email, telephone
			number, job, marital status and fiscal code. 
		\item Third parties' credentials: information that a company has to provide to 					become a registered one: company name, p.iva.
		\item Race information: all the information about the run such as name, date, promoters, 				maximum number of participants and race path.
		\item Partner Application: Application installed on users' device, not necessarily developed by TrackMe, that is in charge with retrieve location and health status. 
	\end{enumerate}
\end{enumerate}
	
\subsection{Revision History}
... Here you see a subsubsection
\subsection{Reference Documents}
... Here you see a subsubsection
\subsection{DocumentStructure}
This document is composed by 6 sections:
\bigbreak
\noindent
\textbf{Section 1} gives an introduction to the problem and describes the purposes of the services-to-be provided by TrackMe. The scope of the application is defined by describing the application domain and listing the goals.
\bigbreak
\noindent
\textbf{Section 2} presents the overall description of the project. Product perspective subsection presents the class diagram describing the domain model used by all the three services. In addition, that subsection include a state diagram that analyzes the process of making a request to access to the users data. User characteristics subsection list the actors interested in using these services.
\bigbreak
\noindent
\textbf{Section 3} specifics the requirements identified, both functional and non functional. The first subsection includes the external interface requirements, showing user interfaces with several mockups. Some scenarios describing specific situations are then listed here. The functional requirements are defined by using use case and sequence diagram. The non functional requirements are defined through performance requirements, design constraints and software system attributes.
\bigbreak
\noindent
\textbf{Section 4} includes the alloy model and the discussion of its purpose. Also, a world generated by it is shown.
\bigbreak
\noindent
\textbf{Section 5} shows the effort spent by each group member while working on this project.
\bigbreak
\noindent
\textbf {Section 6} includes the reference documents
\clearpage
%what you write here is a comment that is not shown in the final text