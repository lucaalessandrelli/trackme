\subsection{Purpose} 
The following Requirements Analysis and Specification Document examines a possible solution for a specific system-to-be provided by the TrackMe company. Therefore, this document contains the description of the scenarios, the use cases that described them, and the models describing requirements and specification for the system-to-be.
\bigbreak
\noindent
Data4Help is a location-based health information service-to-be that allows third parties to monitor the location and health status of individuals. The given problem is to design and develop this service and other two services, AutomatedSOS and Track4Run, which exploit the features offered by Data4Help.
\bigbreak
\noindent
AutomatedSOS is a service-to-be thought to help elderly people. Constantly monitoring the health status of the subscribed customers, this service sends to the user's location an ambulance as soon as the recorded values are anomalous, for example when some health parameters are below certain thresholds.
\bigbreak
\noindent
Finally, Track4Run is a service-to-be that tracks athletes participating in a run. The service, allows organizers to define the path for the run, participants to enroll to the run and spectators to see on a map the position of all the runners during the run.
\subsection{Scope}
\subsubsection{Goals}

\begin{enumerate}
\item[•] {\Large Data4Help}
	\begin{enumerate}
		\item [G.1] Collect users' position and health status.
		\item [G.2] Provide to Third Parties, the users' position and heath status.
		\begin{enumerate}
		\item [G.2.1] Provide data on-demand to non-subscribed third parties.
		\item [G.2.2] Provide data in real-time to subscribed third parties.
		\end{enumerate}
		\item [G.3] Allow third parties two different ways to get users' data.
		\begin{enumerate}
		\item [G.3.1] Allow third parties to get data of a single person.
		\item [G.3.2] Allow third parties to get data of a group of people.
		\end{enumerate}
		\item [G.4] Provide data in an anonymous way, to protect users' privacy.
	\end{enumerate}
	
\item[•] {\Large AutomatedSOS}
	\begin{enumerate}
		\item [G.5] Retrieve user's position and health status.
		\item [G.6] Allow health-interested third parties the access to data detected by AutomatedSOS.
		\item [G.7] Monitor user's health parameters.
		\item [G.8] Send an ambulance to users' location whenever certain parameters are below the threshold.
		\end{enumerate}
	
\item[•] {\Large Track4Run}	
	\begin{enumerate}
		\item [G.5] Retrieve user's position and health status.
		\item [G.9] Allow promoters to manage a run.
		\begin{enumerate}
		\item [G.9.1] Allow promoters to define a path for the run. 
		\item [G.9.2] Allow promoters to invite athletes to the run. 
		\end{enumerate}
		\item [G.10] Allow athletes to enroll on a specific run.
		\item [G.11] Allow spectators to watch in real time the position of every athletes in a specific run.
	\end{enumerate}
\end{enumerate}

\subsubsection{World Phenomena}
... what are world phenomena???

\subsection{Definitions, Acronyms, Abbreviations}

\begin{enumerate}
\item[•] {\Large Definitions}
	\begin{enumerate}
		\item Single request: request of data from a specific registered individual.
		\item Group request: request of data from many individuals. 
		\item Live acquisition: third parties can access to data as soon they are ready, 				through service updates.
		\item On demand acquisition: third parties can access to data when they request 				them.
		\item Subscribers: third parties allowed to receive live acquisition about 						preselected	user/group.
		\item User credentials: information that an individual has to provide to become a 				registered user: name, surname, date of birth, address, email, telephone
			number, job, marital status and fiscal code. 
		\item Third parties' credentials: information that a company has to provide to 					become a registered one: company name, p.iva.
		\item Race information: all the information about the run: name, date, promoters, 				maximum number of participants and race path.
	\end{enumerate}
\end{enumerate}
	
\subsection{Revision History}
... Here you see a subsubsection
\subsection{Reference Documents}
... Here you see a subsubsection
\subsection{DocumentStructure}
... Here you see a subsubsection

%what you write here is a comment that is not shown in the final text