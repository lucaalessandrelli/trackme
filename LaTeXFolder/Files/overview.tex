Here you can see how to include an image in your document.



Here is the command to refer to another element (section, figure, table, ...) in the document: \emph{As discussed in Section~\ref{sect:overview} and as shown in Figure~\ref{fig:metamodel}, ...}. Here is how to introduce a bibliographic citation~\cite{DAM}. Bibliographic references should be included in a \texttt{.bib} file. 

Table generation is a bit complicated in Latex. You will soon become proficient, but to start you can rely on tools or external services. See for instance this \href{https://www.tablesgenerator.com}{https://www.tablesgenerator.com}. 
\subsection{Product perspective}
Here we include further details on the shared phenomena and a domain model (class diagrams and statecharts)
\subsection{Product functions}
Here we include the most important requirements
\subsection{User characteristics}
Here we include everything that is relevant to classify their needs.
\subsection{Assumptions, dependencies and constraints}
In the specification document certain parts were not specific and were ambiguous. So we decided to make the following assumptions.
\subsubsection{Text Assumptions}

\begin{enumerate}
\item[•] {\Large Data4Help}
	\begin{enumerate}
	\item Individuals can always see all the data that have been acquired from his monitoring.
	\item Only third parties can request monitoring service.
	\item Groups are characterized by its member’s attributes (age, gender, city, etc…)
	\item In order to perform single mode acquisition, third parties has to insert fiscal code of tracked user (aka: nor security number nor fiscal code are visible on the application).
	\item In order to perform group mode acquisition, third parties has to select attributes of individuals in which they are inserted 
	\item Discriminative attributes are all the credentials inserted by user (i.e. age, job etc...), their location more or less precise (country, region, city, neighbourhood ..) and the period of time interested (days, weeks, months..)
	\item Third parties in group mode are interested in number of users that match attribute specified, their statistics and the statistics of group health status.
	\item Third parties in single mode are interested to retrieve the sequence of position and  health status information, detected from a certain users during a selected time period.
	\end{enumerate}
	
\item[•] {\Large AutomatedSOS}
	\begin{enumerate}
	\item For this service individuals subscribe to Third Parties and not the other way around.
    \item Third parties that want to exploit this service need to enable the individual registration function.
    \item Who are these Third Parties? (Only croce rossa like , or even doctors…) (dal testo rileggendolo si capisce che sono solo tipo croce rossa)
    \item There are multiple parameters that can have a threshold and every third parties have their own.
    \item Only for elderly people like written in the document or any individual could use this AutomatedSOS
    \item Special devices for elderly people?	
	\end{enumerate}
\item[•] {\Large Track4Run}
	\begin{enumerate}
	\item Any user can organize an event.
    \item An event can be public or private.
    \item An event can have some parameters, for example if it’s public then a parameter could be the maximum amount of participants.
    \item If the event is private then, the organizer need to know the security number or the fiscal code of the athletes to invite them to the event, and also the ones of the spectators.
    \item If the event is public then every user can spectate the event.
    \item All users invited to an event can accept or discard the request.
    \item Race path are always composed by citizen routes (never in private circuits or stadium)
    \end{enumerate}
\end{enumerate}

\subsubsection{Domain Assumptions}

\begin{enumerate}
\item[•] {\Large Data4Help}
	\begin{enumerate}
	\item [D.1.1] The identification (fiscal code, social security number) and the secondary data (attributes) given by the individual during the registration are correct.
    \item [D.1.2] Individuals always have a device with them so that they will be properly monitored.
    \item [D.1.3] The location of individuals is acquired using GPS technology.
    \item [D.1.4] Devices used to monitor individuals always work and report the correct values.
	
	\end{enumerate}
	
\item[•] {\Large AutomatedSOS}
	\begin{enumerate}
	\item [D.2.1] All devices used to monitor the health of the individual always work and report the correct values.
    \item [D.2.2] The ambulance successfully reach the location of the individual.
    \item [D.2.3] The ambulance always get to the location in the minimum amount of time.
    \item [D.2.4] As soon as the parameters get below the threshold, the ambulance gets notified

	\end{enumerate}
\item[•] {\Large Track4Run}
	\begin{enumerate}
	\item [D.3.1] The path defined by the organizer actually exist
    \item [D.3.2] If an athlete enroll to a run then he also participates to the run.
    \item [D.3.3] All athletes have their tracking devices with them for the entire duration of the run.
    \item [D.3.4] Athletes never go out of the defined path defined.
	\end{enumerate}
	
\end{enumerate}