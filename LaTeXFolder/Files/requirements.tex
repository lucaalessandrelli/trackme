\subsection{External Interface Requirements}
\subsubsection{User Interfaces}
\begin{enumerate}
\item[•]{\Large Data4Help}
\bigbreak
\noindent
The third parties interested in having Location and Health Status information of individuals can made the request on the Data4Help’s Website. Since the individuals do not need any particular Data4Help’s App for their data retreivement, Data4Help does not offer any other user interface besides its Website. 
On the Website, thought for the third parties, it is possible both to made group and individual request and to view all the data provided to the specific third party.
\bigbreak
\noindent
The following mockups represent a basic idea of what the Data4Help’s Website will look like in the first release.
\\ [2cm]
\begin{figure}[H]
\centering
\includegraphics[scale=0.3]{Images/Mockups/GR.jpg}
\caption{Group monitoring request}
\end{figure}
\begin{figure}
\centering
\includegraphics[scale=0.3]{Images/Mockups/IR.jpg}
\caption{Individual monitoring request}
\end{figure}
\newpage
\item[•]{\Large AutomatedSOS}
\bigbreak
\noindent
TrackMe offers to AutomatedSOS users an App for smartwatches, with which the users can see their Location and Health Status information. None interface is offered to the third parties since they interact exclusively with Data4Help's Interface. 
\begin{figure}[H]
\begin{center}
        \begin{minipage}[c]{.40\textwidth}
        \centering
          \includegraphics[height=9cm]{Images/Mockups/AutomatedSOSMockup1.png}
	\caption{Welcome page that the user seen in the first App access. Sign up and Sign in are two possibilites }
        \end{minipage}%
        \hspace{10mm}%
        \begin{minipage}[c]{.40\textwidth}
        \centering
          \includegraphics[height=9 cm]{Images/Mockups/AutomatedSOSMockup2.png}
	\caption{Privacy policy conditions regarding Data4Help's treatment of data and group monitoring request}
        \end{minipage}
      \end{center}
\end{figure}
\begin{figure}[H]
\begin{center}
        \begin{minipage}[c]{.40\textwidth}
	\centering
          \includegraphics[height=9 cm]{Images/Mockups/AutomatedSOSMockup3.png}
	\caption{Privacy policy conditions regaring individual minitor request, the user can agree to these terms or not. }
        \end{minipage}%
        \hspace{10mm}%
        \begin{minipage}[c]{.40\textwidth}
	\centering
          \includegraphics[height=9 cm]{Images/Mockups/AutomatedSOSMockup4.png}
	\caption{Message that commuincates to the user the importance of ever wearing the smartwatch}
        \end{minipage}
      \end{center}
\end{figure}
\begin{figure}[H]
\begin{center}
        \begin{minipage}[c]{.35\textwidth}
	\centering
          \includegraphics[height=9 cm]{Images/Mockups/AutomatedSOSMockup5.png}
	\caption{Main menu showing the various functions offered to user}
        \end{minipage}%
        \hspace{10mm}%
        \begin{minipage}[c]{.40\textwidth}
	\centering
          \includegraphics[height=9 cm]{Images/Mockups/AutomatedSOSMockup6.png}
	\caption{Fields the user have to fill to complete the registration}
        \end{minipage}
      \end{center}
\end{figure}
\begin{figure}[H]
\begin{center}
        \begin{minipage}[c]{.35\textwidth}
	\centering
          \includegraphics[height=9cm]{Images/Mockups/AutomatedSOSMockup7.png}
	\caption{Warning message that comunicates to the user that an ambulance has been sent to the user location}
        \end{minipage}%
        \hspace{10mm}%
        \begin{minipage}[c]{.40\textwidth}
	\centering
        \end{minipage}
      \end{center}
\end{figure}
\item[•]{\Large Track4Run}
\bigbreak
\noindent
Track4Run users can use an App for smartphone and another one for smartwatches. The first one could be used by everyone, while the second one is made only for the athletes. Like for AutomatedS0S, there is not any interface provided for the third parties.
\bigbreak
\noindent
The mockups showed in the next page represent a basic idea of what tthe Track4Run's App for smartphone will look like in the first release.
\\[6cm]
\begin{figure}[H]
\begin{center}
        \begin{minipage}[c]{.40\textwidth}
        \centering
          \includegraphics[height=10.3 cm]{Images/Mockups/Track4RunMockup1.jpg}
	\caption{Welcome page }
        \end{minipage}%
        \hspace{10mm}%
        \begin{minipage}[c]{.40\textwidth}
        \centering
          \includegraphics[height=10.3 cm]{Images/Mockups/Track4RunMockup2.jpg}
	\caption{Privacy policy conditions pt.1}
        \end{minipage}
      \end{center}
\end{figure}
\begin{figure}[H]
\begin{center}
        \begin{minipage}[c]{.40\textwidth}
        \centering
          \includegraphics[height=10.3 cm]{Images/Mockups/Track4RunMockup3.jpg}
	\caption{Privacy policy conditions pt.2 }
        \end{minipage}%
        \hspace{10mm}%
        \begin{minipage}[c]{.40\textwidth}
        \centering
          \includegraphics[height=10.3 cm]{Images/Mockups/Track4RunMockup4.jpg}
	\caption{Registration form}
        \end{minipage}
      \end{center}
\end{figure}
\begin{figure}[H]
\begin{center}
        \begin{minipage}[c]{.40\textwidth}
        \centering
          \includegraphics[height=10.3 cm]{Images/Mockups/Track4RunMockup5.jpg}
	\caption{Main user menu}
        \end{minipage}%
        \hspace{10mm}%
        \begin{minipage}[c]{.40\textwidth}
        \centering
          \includegraphics[height=10.3cm]{Images/Mockups/Track4RunMockup6.jpg}
	\caption{Promote a run view}
        \end{minipage}
      \end{center}
\end{figure}
\begin{figure}[H]
\begin{center}
        \begin{minipage}[c]{.40\textwidth}
        \centering
          \includegraphics[height=10.3cm]{Images/Mockups/Track4RunMockup7.jpg}
	\caption{Enroll to a run view}
        \end{minipage}%
        \hspace{10mm}%
        \begin{minipage}[c]{.40\textwidth}
        \centering
          \includegraphics[height=10.3 cm]{Images/Mockups/Track4RunMockup8.jpg}
	\caption{Spectate a run view}
        \end{minipage}
      \end{center}
\end{figure}
\end{enumerate}

\subsubsection{Hardware and Software Interfaces}
None of the three services-to-be offer any hardware or software interfaces to external world.
\subsubsection{Communication Interfaces}
\subsection{Scenarios}
\begin{enumerate}
\item[•]{\Large Scenario 1} \\
The company StatisticsDispenser is interested into weekly providing public statistics about people living in London. For this reason the company, which is already registered to Data4Help, need to send a group monitoring request. After logging into the Data4Help website, StatisticsDispenser open the group request section. The website loads a new page where the company can filter groups through some attributes regarding his members like the age, the gender, the city and many more. For the specific purpose StatisticsDispenser chooses only to filter people who live in London and people who's age is between 20 and 60. Then, due to the fact that the company need future data, StatisticsDispenser subscribes to the group. From now on every time new data is available the system sends a notification to StatisticsDispencer. 
\item[•]{\Large Scenario 2} \\
Mark often goes for a run so he decides to download an app to track his progress. The app he installed on his smartphone is a Partner Application with Data4Help. After he registers to this app he's asked if he wants to give his information to the company TrackMe and also if he wants his data to be treated also as individual data. Right after he accepts the policy he's asked to create a Data4Help account or to link an existing one to the application. Mark never created a Data4Help account so he decides to go through the registration process. He fills all the attributes fields required for the sign up and confirm the registration. Data4Help creates the account and saves all the attributes that Mark filled in. Data4Help is now ready to receive Mark's data from the partner application.
\item[•]{\Large Scenario 3} \\
Bob is 77 years old and lately he's having heart problems. Under his son's advice he decided to use the AutomatedSOS application to receive immediate aid in case of need. One month later Bob doesn't feel okay and his heartbeat value goes below the threshold. AutomatedSOS recognizes that Bob is in a critical health state and quickly sends a report to First Aid system containing useful information like his current location and the reason he's in danger. After the First Aid system received the report, it immediately sends an ambulance to Bob's location and then sends an acknowledge message to AutomatedSOS. The application shows a message on Bob's device to let him know that an ambulance is coming for aid.
\item[•]{\Large Scenario 4} \\
Mario promotes run events for a living so in order to simplify the process he goes through everyday he decides to download Track4Run on his smartphone. The famous company AdiDas designate Mario to promote a run that takes place once a year in Milan. Mario log into the app and enter the "Promote a Run" section, inserts and confirms all the information needed. Track4Run creates the run and makes it available to athletes to enroll in.
\item[•]{\Large Scenario 5} \\
Lately Eddie and his friends are bored of what they usually do so they decided to participate to a different activity. More precisely they want to create a run event using Track4Run App and see who's the fastest at running. Unfortunately Eddie got ill the day before the event but he's just too curios of seeing who of his friends is going to win. For this reason as soon as the run starts he logs into Track4Run and enters the "Spectate a Run" section and select the run created with his friends. A few moments later the map appears on the app with all the athletes positions on it letting Eddie see how the run is proceeding in real time.

\end{enumerate}

\subsection{Functional Requirements}
\begin{enumerate}
\item[•]{\Large Data4Help}
	\begin{enumerate}
	\item [G.1] \textbf{Collect users' position and health status.}
		\begin{enumerate}
		\item [D.1] Users' information are collected from partner applications or from the other two TrackMe applications installed on users' devices.
		\item [D.2] All the partner applications require to submit user credentials.
		\item [D.3] The identification (fiscal code, social security number) and the secondary data (attributes) given by the individual during the registration are correct.
		\item [R.1] Retrieve user credentials inserted into partner application as group attributes.
		\item [R.2] Allow users already registered in Data4Help world to sign in with his account without provide user credentials again.
		\item [R.3] Allow individuals to agree with privacy policy (first part). Registered users, now, can be tracked in group mode request through installed application.  
		\item [R.4] Allow individuals to specify, during registration, if they are also interested to be tracked in single mode request (second part) through installed application.
		\item [R.5] After registration the system provides, through provided e-mail, the password to access the data.
		\item [D.4] Devices used to monitor individuals always work and report 			the correct values.	
    	\item [D.5] Partner application always report correct values to Data4Help.
    	\item [R.6] The system has to correctly receive data from partner applications installed on users' device.
    	\end{enumerate}	
    	
    \item [G.2] \textbf{Provide to Third Parties, the users' position and heath status.}
    	\begin{enumerate} 
    	\item [R.7] Allow third parties registration to Data4Help service, where they have to specify all their credentials.
    	\item [R.5] After registration the system provides, through provided e-mail, the password to access the data.
    	\end{enumerate}	
		
		\begin{enumerate} 
		\item [G.2.1] \textbf{Provide data on-demand to non-subscribed third parties.}
		\begin{enumerate} 
		\item [R.8] For each user registered ,the system has to automatically retrieve and store data from partner applications with a resolution of 10 minutes; independently from the requests reached.	
		\item [R.9] The system has to collect inside his database all the useful information that match the request.
		\item [R.10] The system has to send to applicant all the data already collected.
    	\end{enumerate}	
    	
    	\item [G.2.2] \textbf{Provide data in real-time to subscribed third parties.}
		\begin{enumerate}
		\item [D.8] Live acquisition lasts 24 hours to reduce waste of resources.
    	\item [R.11] Allow third parties subscription to interested group in order to receive live data.
    	\item [R.12] When a real time request is performed the system has to collect and store specific users' data with a resolution of 2 minutes.
    	\item [R.13] Provide to subscribed third parties data as soon as they are available by the system.
    	\end{enumerate}
    	\end{enumerate}
    
	\item [G.3] \textbf{Allow third parties two different ways to get users' data.}
		\begin{enumerate}     
    	\item [G.3.1] \textbf{Allow third parties to get data of a single person.}
		\begin{enumerate}
		\item [D.6] In order to perform an individual request, third parties has to know the user's fiscal code or security number.
		\item [D.7] Security number and fiscal code are not information given to third parties by Data4Help.
    	\item [R.14] Allow third parties to insert fiscal code of user that want to track.
    	\item [R.15] Deny third parties to receive information about users in  single mode, that have not accepted second part of privacy policy.
    	\item [R.16] Collect all the useful information retrieved by Data4Help that are produced by the interested users 
    	\item [R.17] Send all the collected information to request applicant.
    	\end{enumerate}
    
    	\item [G.3.2] \textbf{Allow third parties to get data of a group of people.}
		\begin{enumerate}
    	\item [R.18] Allow third parties to insert attributes in which they are interested to restrict their field of search.
    	\item [R.19] Deny third parties to receive information if the provided information can hurt users' privacy, for this purpose group request under 1000 users involved are rejected.
    	\item [R.16] Collect all the useful information retrieved by Data4Help that are produced by the interested users 
    	\item [R.17] Send all the collected information to request applicant.
    	\end{enumerate}
    	\end{enumerate}
    	
    \item [G.4] \textbf{Provide data in an anonymous way, to protect users' privacy.}
		\begin{enumerate}
    	\item [R.15] Deny third parties to receive information about users in  single mode, that have not accepted second part of privacy policy.
    	\item [R.19] Deny third parties to receive information if the provided information can hurt users' privacy, for this purpose group request under 1000 users involved are rejected.
    	\end{enumerate}	
			
	\end{enumerate}
	
	
\item[•]{\Large AutomatedSOS}
	
	\begin{enumerate}
	\item [G.5] \textbf{Retrieve user's position and health status.}
		\begin{enumerate}
		\item [R.20] Allow users to be tracked from AutomatedSOS filling up the registration and agreeing to both parts of privacy policy.
		\item [D.4] Devices used to monitor individuals always report correct values.
		\item [D.9] The user always dresses a smartwatch on which AutomatedSOS is installed.    
		\item [R.21] The application has to interact with Smartwatch/Smartphone APIs in order to retrieve location and health status.
		\item [R.22] The application is able to send to Data4Help service all the informations already retrieved in live acquisition.
		\end{enumerate}
		
	\item [G.6] \textbf{Allow health-interested third parties the access to data detected by AutomatedSOS.}
		\begin{enumerate}
		\item [R.23] Allow non-profit organizations to register into AutoatedSOS portal and becoming health third parties.
		\item [R.24] Allow health third parties to receive informations about all the users registered to AutomatedSOS through Live Acquisition performed by Data4Help.
		\end{enumerate}
	
	\item [G.7] \textbf{Monitor user's health parameters.}
		\begin{enumerate}
		\item [R.21] The application has to interact with Smartwatch/Smartphone APIs in order to retrieve location and health status.
		\item [R.25] The application has to retrieve users' health status every 2 seconds in order to guarantee reaction time of 5 seconds.
		\end{enumerate}
		
	\item [G.8] \textbf{Send an ambulance to users' location whenever certain parameters are below the threshold.}
		\begin{enumerate}
		\item [R.26] The application has to control health status with data retrieved in local to realize immediately if certain parameters are critical.
		\item [R.27] The application has to call an ambulance, if parameters are critical.
		\item [D.10] The ambulance system is always up and ready to receive messages from AutomatedSOS.
		\item [R.28] Supply to hospital user's location and all the useful information to provide efficient first aid.
		\item [R.29] If none answer arrives from the hospital the software must repeat another time the request until an answer in reached.
		\item [D.11] The ambulance successfully reach the location of the individual.
		\end{enumerate}
		
  	\end{enumerate}
  	
  	
\item[•]{\Large Track4Run}
	
	\begin{enumerate}
	\item [G.5] \textbf{Retrieve user's position and health status.}
		\begin{enumerate}
		\item [R.30] Allow users to be tracked from Track4Run filling up the registration and agreeing to both parts of privacy policy.
		\item [D.4] Devices used to monitor individuals always report correct values.
		\item [R.21] The application has to interact with Smartwatch/Smartphone APIs in order to retrieve location and health status.
		\item [R.22] The application is able to send to Data4Help service all the informations already retrieved in live acquisition.
		\end{enumerate}
		
	\item [G.9] \textbf{Allow promoters to manage a run.}
		\begin{enumerate}
		\item [R.31] Allow users to create a run providing all the general information about the competition.
		\item [R.32] Allow users to specify if the race is public or private.
		\item [D.4] Devices used to monitor individuals always report correct values.
		\item [D.13] During a run athletes always dress a smartwatch on which Track4Run is installed.
			
		\item [G.9.1] \textbf{Allow promoters to define a path for the run.}
			\begin{enumerate}
			\item [R.33] Allow promoters to define a path for the race by selecting the routes inside a map.
			\item [D.14] The path defined by the organizer actually exist.
			\end{enumerate}
			
		\item [G.9.2] \textbf{Allow promoters to invite athletes to the run.}
			\begin{enumerate}
			\item [R.34] Allow promoters to invite athlete to be runner of their private race.
			\item [R.35] Allow promoters to specify maximum number of athletes that can take part to their public race.
			\end{enumerate}
	\end{enumerate}
	
	\item [G.10] \textbf{Allow athletes to enroll on a specific run.}
		\begin{enumerate}
		\item [R.36] Allow users to see all the public races and private races in which he is invited.
		\item [R.37] Allow user to select a race and add him to the athletes involved.
		\item [D.16] If an athlete enroll to a run then he also participates to the run.
		\end{enumerate}
	
	\item [G.11] \textbf{Allow spectators to watch in real time the position of every athletes in a specific run.}
		\begin{enumerate}
		\item [D.17] All athletes have their tracking devices with them and the application enabled for the entire duration of the run.	
		\item [R.38] Allow user to select a race to be viewed.
		\item [R.39] The application must be able to request to Data4Help the positions of all the other athletes involved.
		\item [R.40] The application must be able to receive and display the positions of all the other athletes involved.
		\item [D.18] Athletes never go out of the defined path.
		\end{enumerate}
	\end{enumerate}

\end{enumerate}


\subsubsection{Use Case Diagram}
\begin{enumerate}
\begin{minipage}{\textwidth}
\FloatBarrier
\item[•]{\Large Data4Help}

\begin{center}
\includegraphics[scale=0.65]{Images/UseCaseDiagrams/Data4HelpUseCaseDiagram.png}
\end{center}
\FloatBarrier
\end{minipage}


\begin{minipage}{\textwidth}
\item[•]{\Large AutomatedSOS}
\FloatBarrier
\begin{center}
\includegraphics[scale=0.65]{Images/UseCaseDiagrams/AutomatedSOSCaseDiagram.png}
\end{center}
\FloatBarrier
\end{minipage}

\begin{minipage}{\textwidth}
\item[•]{\Large Track4Run}
\FloatBarrier
\begin{center}
\includegraphics[scale=0.70]{Images/UseCaseDiagrams/Track4RunCaseDiagram.png}
\end{center}
\FloatBarrier
\end{minipage}
\end{enumerate}

\subsubsection{Use Cases}
\begin{enumerate}
\FloatBarrier
\item[•]{\Large Data4Help Use Cases}
\FloatBarrier
\begin{table}[h]
\begin{tabular}{|l|p{.75\textwidth}|}
\hline
Name             & Sign Up \\ \hline
Actors           & Third Party \\ \hline
Entry Conditions & TRUE    \\ \hline
Event Flow       & \begin{enumerate}
            \item The Third Party enters the sign up section.
            \item The system shows to the third party all the attributes fields needed for the registration.
            \item The Third Party fills all the attribute fields.
            \item The Third Party confirms all the attributes inserted stating he wants to register.
            \item The system creates and saves the third party's account.
        \end{enumerate}\\ \hline
Exit Condition   & The third party's account has been created and the third party is now registered.\\ \hline
Exceptions       & \begin{itemize}
\item If the system notices that the attributes used in the registration are already linked to an existing account then a warning is generated saying that there is already a third party registered with the given attributes.
\end{itemize}\\ \hline
\end{tabular}
\end{table}
\FloatBarrier

\FloatBarrier
\begin{table}[h]
\begin{tabular}{|l|p{.75\textwidth}|}
\hline
Name             & Sign In \\ \hline
Actors           & Third Party  \\ \hline
Entry Conditions & TRUE   \\ \hline
Event Flow       & \begin{enumerate}
            \item The Third Party enters the sign in section.
            \item The system shows to the third party all the credentials fields needed for the log in.
            \item The Third Party fills all credentials fields and confirms he wants to log in.
            \item The system accepts the log in request.
        \end{enumerate}\\ \hline
Exit Condition   & The third party is now logged in.\\ \hline
Exceptions       & \begin{itemize}
\item If the third party inserts invalid log in credentials a warning is generated saying the credentials are invalid.
\end{itemize}\\ \hline
\end{tabular}
\end{table}
\FloatBarrier

\FloatBarrier
\begin{table}[h]
\begin{tabular}{|l|p{.75\textwidth}|}
\hline
Name             & Request Individual Monitoring \\ \hline
Actors           & Third Party  \\ \hline
Entry Conditions & The third party is logged in.    \\ \hline
Event Flow       & \begin{enumerate}
            \item The Third Party enters the Individual Request section.         
            \item The system shows to the third party all the information fields needed for the identification of the individual.
            \item The Third Party fills all the attribute fields and confirms he wants to track that specific individual.
            \item The system shows all the individual's information that have been collected until the moment of the request. 
        \end{enumerate}\\ \hline
Exit Condition   & The request's outcome is shown to the third party.\\ \hline
Exceptions       & \begin{itemize}
\item If the inserted attributes are not linked to any user account then a warning message is displayed saying that the individual is not registered.
\item If the individual that correspond to the attributes inserted didn't accept the individual treatment of data policy then a warning message is displayed saying that the request is rejected.
\end{itemize}\\ \hline
\end{tabular}
\end{table}
\FloatBarrier

\FloatBarrier
\begin{table}[h]
\begin{tabular}{|l|p{.75\textwidth}|}
\hline
Name             & Request Group Monitoring \\ \hline
Actors           & Third Party  \\ \hline
Entry Conditions & The third party is logged in.    \\ \hline
Event Flow       & \begin{enumerate}
            \item The Third Party enters the Group Request section.
            \item The system shows to the third party all the information fields needed for defining the group.
            \item The Third Party inserts all the attributes.
            \item The system accepts the request.
            \item The system shows all the group's information that have been collected until the moment of the request. 
        \end{enumerate}\\ \hline
Exit Condition   & The request's outcome is shown.\\ \hline
Exceptions       & \begin{itemize}
\item If the group request get rejected by the system a warning message will be displayed saying the request is rejected.
\end{itemize}\\ \hline
Special Requirements & The system rejects group monitoring requests when the group's information can compromise users' privacy. For this purpose requests of groups composed by less than 1000 users get rejected.
\\ \hline
\end{tabular}
\end{table}
\FloatBarrier

\FloatBarrier
\begin{table}[h]
\begin{tabular}{|l|p{.75\textwidth}|}
\hline
Name             & Subscribe To A Group\\ \hline
Actors           & Third Party  \\ \hline
Entry Conditions & The third party has just sent an accepted monitoring request to a group. \\ \hline
Event Flow       & \begin{enumerate}
            \item The Third Party requests to follow the selected group.
			\item The system links the third party to the group.
            \item The system sends new data to the third party.
        \end{enumerate}\\ \hline
Exit Condition   & The Third Party is subscribed to the selected group.\\ \hline
Exceptions       & \begin{itemize}
            \item If the third party is already subscribed a warning message is shown saying the subscription have been already done.
        \end{itemize}\\ \hline
\end{tabular}
\end{table}
\FloatBarrier

\FloatBarrier
\begin{table}[h]
\begin{tabular}{|l|p{.75\textwidth}|}
\hline
Name             & Sign Up From Partner App\\ \hline
Actors           & User, Partner Application  \\ \hline
Entry Conditions & The user accepted the treatment of data policy.  \\ \hline
Event Flow       & \begin{enumerate}
			\item The user starts the sign up function on the partner app.
			\item The Partner Application shows to the user all the attributes fields needed for the registration.
            \item The User fills all the attribute fields.
            \item The Partner Application sends to the system the attributes inserted by the user.
            \item The system receives by the partner application all the attributes inserted by the user.
            \item The system creates the user's account and saves the received data.
        \end{enumerate}\\ \hline
Exit Condition   & The system registered the user.\\ \hline
Exceptions       & \begin{itemize}
\item If the system notices that attributes used in the registration are already linked to an existing account then a message is sent back to the partner application in order to let the user know what happened.
\end{itemize}\\ \hline
\end{tabular}
\end{table}
\FloatBarrier

\FloatBarrier
\begin{table}[h]
\begin{tabular}{|l|p{.75\textwidth}|}
\hline
Name             & Link Account To The Partner App\\ \hline
Actors           & User, Partner Application  \\ \hline
Entry Conditions & The user accepted the treatment of data policy and already has an existing account to link to the partner application.  \\ \hline
Event Flow       & \begin{enumerate}
			\item The user starts the account linking function on the partner app.
			\item The Partner Application shows to the user all the credential fields needed for the linking process.
            \item The User fills all the credential fields.
            \item The Partner Application sends to the system the credentials inserted by the user.
            \item The system receives by the partner application all the credentials inserted by the user.
            \item The system sends back to the partner application the outcome of the operation. 
        \end{enumerate}\\ \hline
Exit Condition   & The system registered the user.\\ \hline
Exceptions       & \begin{itemize}
\item If the system notices that the credentials received are not linked to an existing account then a message is sent back to the partner application in order to let the user know what happened.
\end{itemize}\\ \hline
\end{tabular}
\end{table}
\FloatBarrier


\FloatBarrier
\begin{table}[h]
\begin{tabular}{|l|p{.75\textwidth}|}
\hline
Name             & Send User's Data\\ \hline
Actors           & User, Partner Application  \\ \hline
Entry Conditions & The user accepted the treatment of data policy and the partner application is running on the user's device.  \\ \hline
Event Flow       & \begin{enumerate}
			\item The Partner Application collects user's data.
            \item The Partner Application sends the user's data to the the system.
            \item The system receives and saves the user's data sent by the partner application.
        \end{enumerate}\\ \hline
Exit Condition   & The system saved the user's data.\\ \hline
Exceptions       & None \\ \hline
\end{tabular}
\end{table}
\FloatBarrier

\item[•]{\Large AutomatedSOS Use Cases}
\FloatBarrier
\begin{table}[h]
\begin{tabular}{|l|p{.75\textwidth}|}
\hline
Name             & Sign Up \\ \hline
Actors           & User  \\ \hline
Entry Conditions & The User has AutomatedSOS installed on his smartwatch.    \\ \hline
Event Flow       & \begin{enumerate}
            \item The User enters the sign up section of the app.
            \item The User accepts the treatment of data policy.
            \item The system shows to the user all the attributes fields needed for the registration.
            \item The User fills all the attribute fields.
            \item The User confirms all the attributes inserted stating he wants to register.
            \item The system creates and saves the user's account.
        \end{enumerate}\\ \hline
Exit Condition   & The user's account has been created and the user is now registered.\\ \hline
Exceptions       & \begin{itemize}
\item If the User does not accept the treatment of data policy then a warning is generated saying that ,in order to register, the policy must be accepted.
\item If the system notices that attributes used in the registration are already linked to an existing account then a warning is generated saying that there is already an individual registered with the given credentials.
\end{itemize}\\ \hline
\end{tabular}
\end{table}
\FloatBarrier

\FloatBarrier
\begin{table}[h]
\begin{tabular}{|l|p{.75\textwidth}|}
\hline
Name             & Sign In \\ \hline
Actors           & User  \\ \hline
Entry Conditions & The User has AutomatedSOS application installed on his smartwatch.    \\ \hline
Event Flow       & \begin{enumerate}
			\item The User enters the sign in section of the app.
            \item The system shows to the user all the credentials fields needed for the log in.
            \item The User fills all credentials fields and confirms he wants to log in.
            \item The User clicks the "Enter" button.
            \item The system accept the log in request.
        \end{enumerate}\\ \hline
Exit Condition   & The User user is now logged in.\\ \hline
Exceptions       & \begin{itemize}
\item If user inserts invalid log in credentials a warning is generated saying the credentials are invalid.
\end{itemize}\\ \hline
\end{tabular}
\end{table}
\FloatBarrier

\FloatBarrier
\begin{table}[h]
\begin{tabular}{|l|p{.75\textwidth}|}
\hline
Name             & See Acquired Data \\ \hline
Actors           & User  \\ \hline
Entry Conditions & The User is logged in. \\ \hline
Event Flow       & \begin{enumerate}
            \item The User enters the Acquired Info section of the app.
            \item The system gets all the user's information that have been retrieved by the application until that moment.
            \item The system displays the user's information.
\end{enumerate}\\ \hline
Exit Condition   & All the information retrieved by the system are shown on the app.\\ \hline
Exceptions       & \begin{itemize}
\item If the system do not find information about the user then a warning message is shown to the user saying that until now the application did not record any information.
\end{itemize}  \\ \hline
\end{tabular}
\end{table}
\FloatBarrier

\FloatBarrier
\begin{table}[h]
\begin{tabular}{|l|p{.75\textwidth}|}
\hline
Name             & Set Preferences \\ \hline
Actors           & User  \\ \hline
Entry Conditions & The User is logged in. \\ \hline
Event Flow       & \begin{enumerate}
            \item The User enters the Preferences section of the application.
            \item The User can add or remove certain health parameters in order to personalize the monitoring profile.
            \item The User can also change certain parameters threshold.
            \item The User confirms all the changes done.
            \item The system saves all the changes made by the user.
        \end{enumerate}\\ \hline
Exit Condition   & The parameters are correctly updated as the user wants them to be.\\ \hline
Exceptions       & \begin{itemize}
\item If the user does not confirm the changes then all parameters remain the same as before. 
\end{itemize} \\ \hline
\end{tabular}
\end{table}
\FloatBarrier

\FloatBarrier
\begin{table}[h]
\begin{tabular}{|l|p{.75\textwidth}|}
\hline
Name             & Send Ambulance Request \\ \hline
Actors           & AutomatedSOS, First Aid  \\ \hline
Entry Conditions & A critical health parameter value is below the threshold.
\\ \hline
Event Flow       & \begin{enumerate}
            \item AutomatedSOS sends to First Aid a report that contains all important information about the user like his current location, his gender, his age, his health profile, and the list of parameters that got below the threshold.
            \item First Aid immediately sends an ambulance to the user's location.
			\item First Aid sends an acknowledge message to AutomatedSOS.            
            \item AutomatedSOS displays on the app a warning message saying that an ambulance is currently heading to the user's location. 
        \end{enumerate}
\\ \hline
Exit Condition   & A warning message is shown saying that an ambulance is currently heading to the user's location.
\\ \hline
Exceptions       & \begin{itemize}
\item If no acknowledge message is received by AutomatedSOS after the form has been sent, as soon as a certain time out expires AutomatedSOS re-send the form with updated information. 
\end{itemize}
\\ \hline
Special Requirements & The form need to be sent to First Aid with a reaction time of less than 5 seconds from the time the parameters are below the threshold.
\\ \hline 
\end{tabular}
\end{table}
\FloatBarrier

\FloatBarrier
\item[•]{\Large Track4Run Use Cases}
\begin{table}[h]
\begin{tabular}{|l|p{.75\textwidth}|}
\hline
Name             & Sign Up \\ \hline
Actors           & User  \\ \hline
Entry Conditions & The User has Track4Run application installed on his device.    \\ \hline
Event Flow       & \begin{enumerate}
			\item The User enters the sign up section of the app.
            \item The User accepts the treatment of data policy.
            \item The system shows to the user all the attributes fields needed for the registration.
            \item The User fills all the attribute fields.
            \item The User confirms all the attributes inserted stating he wants to register.
            \item The system creates and saves the user's account.
        \end{enumerate}\\ \hline
Exit Condition   & The user's account has been created and the user is now registered.\\ \hline
Exceptions       & \begin{itemize}
\item If the User does not accept the treatment of data policy then a warning is generated saying that ,in order to register, the policy must be accepted.
\item If the system notices that attributes used in the registration are already linked to an existing account then a warning is generated saying that there is already an individual registered with the given credentials.
\end{itemize}\\ \hline
\end{tabular}
\end{table}
\FloatBarrier

\FloatBarrier
\begin{table}[h]
\begin{tabular}{|l|p{.75\textwidth}|}
\hline
Name             & Sign In \\ \hline
Actors           & User  \\ \hline
Entry Conditions & The User has Track4Run application installed on his smartwatch.    \\ \hline
Event Flow       & \begin{enumerate}
			\item The User enters the sign in section of the app.
            \item The system shows to the user all the credentials fields needed for the log in.
            \item The User fills all credentials fields and confirms he wants to log in.
            \item The User clicks the "Enter" button.
            \item The system accepts the log in in request.
        \end{enumerate}\\ \hline
Exit Condition   & The User user is now logged in.\\ \hline
Exceptions       & \begin{itemize}
\item If the user inserts invalid log in credentials a warning is generated saying the credentials are invalid.
\end{itemize}\\ \hline
\end{tabular}
\end{table}
\FloatBarrier

\FloatBarrier
\begin{table}[h]
\begin{tabular}{|l|p{.75\textwidth}|}
\hline
Name             & Promote A Run \\ \hline
Actors           & User  \\ \hline
Entry Conditions & The User is logged in.    \\ \hline
Event Flow       & \begin{enumerate}
            \item The User enters the Promote a Run section of the app.
            \item The system shows to the user a new tab where the user can define all the important information about the run and also invite athletes.
            \item The system creates and saves the run's information.
            \item The system automatically sends notifications to all athletes specified by the promoter asking them if they want to participate to the run.
        \end{enumerate}\\ \hline
Exit Condition   & The run event has been created and added to the list of promoted runs.\\ \hline
Exceptions       & \begin{itemize}
\item If the user does not insert critical information (like the path, the name or the date) a warning message is shown saying that critical parameters are missing.
\end{itemize}\\ \hline
\end{tabular}
\end{table}
\FloatBarrier

\FloatBarrier
\begin{table}[h]
\begin{tabular}{|l|p{.75\textwidth}|}
\hline
Name             & Enroll To A Run \\ \hline
Actors           & User  \\ \hline
Entry Conditions & The User is logged in.    \\ \hline
Event Flow       & \begin{enumerate}
            \item The User enters to the "Enroll to a Run" section of the app.
            \item The system shows to the user the list of all the created runs that will take place in the future.
            \item The User can filter the runs with some attributes.
            \item The User chooses the run he wants to participate to.
            \item The system adds the user to the list of athletes enrolled to the run.
        \end{enumerate}\\ \hline
Exit Condition   & The user is now enrolled to the run.\\ \hline
Exceptions       & \begin{itemize}
\item If the number of athletes has already capped the max amount in the run chosen by the user then a warning message is displayed saying that no more athletes are allowed to participate to the run.
\item If the user didn't accept the treatment of data policy, a warning message is displayed, asking the user to accept it in order to enroll to the run.
\end{itemize}\\ \hline
\end{tabular}
\end{table}
\FloatBarrier

\FloatBarrier
\begin{table}[h]
\begin{tabular}{|l|p{.75\textwidth}|}
\hline
Name             & Spectate A Run \\ \hline
Actors           & User  \\ \hline
Entry Conditions & The User is logged in.    \\ \hline
Event Flow       & \begin{enumerate}
            \item The User enters the Spectate a Run section of the app.
            \item The system shows a new tab where the list of all live runs is visible.
            \item The User can filter the runs with some attributes.
            \item The User chooses the run he wants to spectate.
            \item The system shows to the user the map of the run and also the position of all athletes in real time.
        \end{enumerate}\\ \hline
Exit Condition   & The system is showing to the user the map and the athletes positions.\\ \hline
Exceptions       & None.\\ \hline
\end{tabular}
\end{table}
\FloatBarrier

\end{enumerate}



\subsection{Sequence Diagram}
\begin{enumerate}
\begin{minipage}{\textwidth}
\FloatBarrier
\item[•]{\Large Data4Help}


Individual request with on-demand acquisition performance.
\begin{center}
\includegraphics[scale=0.8]{Images/Seq_Data4Help_onDem.png}
\end{center}
\FloatBarrier

\FloatBarrier
Automatic data update inside Data4Help:
\begin{center}
\includegraphics[scale=0.8]{Images/Seq_Data4Help_autoUp.png}
\end{center}
\FloatBarrier

\end{minipage}

\begin{minipage}{\textwidth}
\FloatBarrier
Group request with live acqusition performance:
\begin{center}
\includegraphics[scale=0.75]{Images/Seq_Data4Help_live.png}
\end{center}

\FloatBarrier
User registration performance:
\begin{center}
\includegraphics[scale=0.75]{Images/Seq_Data4Help_registration.png}
\end{center}


\FloatBarrier
\end{minipage}


\begin{minipage}{\textwidth}
\item[•]{\Large AutomatedSOS}
\FloatBarrier
AutomatedSOS monitor and ambulance caller services:
\begin{center}
\includegraphics[scale=0.8]{Images/Seq_AutoSOS_monitor.png}
\end{center}
\FloatBarrier

\item[•]{\Large Track4Run}
\FloatBarrier
Track4Run automated retrieve athletes' position and update spectators' live map.
\begin{center}
\includegraphics[scale=0.8]{Images/Seq_Track4Run_raceUp.png}
\end{center}
\FloatBarrier
\end{minipage}

\end{enumerate}


\subsection{Performance Requirements}
\begin{enumerate} 
\item[•] Data4Help service is build to perform trend research from users that download specific partner applications, in order to perform this type of monitoring there is a high use of resources (specially in live acquisition when data must be exchanged within 2 minutes) therefore ,initially, this application is developed to track 10.000 users simultaneously (included users from AutomateSOS and Track4Run). To serve third parties is required less performance because they are very few in comparison to users.

\item[•] AutomatedSOS is a very expensive application in terms of performance because it must monitor the individual all the day and guarantee a data collection with an interval of 2 seconds.

\item[•] Track4Run is a variable expensive application because it has to monitor individual constantly during the race, but not all day, all the athletes have a run.
\end{enumerate}

\subsection{Design Constraints}
\subsubsection{Standards compliance}
\begin{enumerate} 
\item[•] Partner applications request the permission to retrieve location and health status to the device, same for AutomateSOS and Track4Run.
\item[•] Data4Help requires that partner applications can use internet connection and users' mobile data to exchange information.
\item[•] AutomatedSOS requires all day internet connection in order to call an ambulance every time is needed.
\item[•] AutomatedSOS requires internet connection during all the duration of the race to the athletes and to the spectator.
\end{enumerate}

\subsubsection{Hardware limitations}
\begin{enumerate} 
\item[•] AutomatedSOS application requires that is installed on a smartwatch (smartphone is not enough) in order to acquire location and health status.
\item[•] Track4Run application requires that is installed on a smartphone (or a smartwatch) in order to acquire location.
\item[•] Smartphone and smartwatch must be iOS or Android platform.
\item[•] The devices must have internet connection (mobile data are mandatory).
\item[•] The devices must have GPS locator.
\item[•] Smartwatch must have Heart Rate monitor, Blood Pressure monitor, Pedometer, Calories Calculator.
\end{enumerate}

\subsection{Software System Attributes}
\subsubsection{Reliability}
Data4Help service,clearly, has some moments with less load (as the night) but is important to guarantee a 24/7 service. Some small concessions are possible during the night.
In order to guarantee AutomatedSOS monitoring this service must be available 24/7, also the night.
Track4Run service has the same consideration of Data4Help.
\subsubsection{Availability}
In order to guarantee high degree of availability, and considering that the main core of the service are data, is necessary high level of data redundancy about.
This system is expected to be available 99.99%. 
\subsubsection{Security}
Security for sensitive informations are one of the goal of this system, in order to guarantee users' privacy, the system implements certificated security communication protocol.
In any cases users can read and agree regulatory policy first.
\subsubsection{Maintainability}
In order to guarantee maintainability the entire software project is based on Data4Help primitives (as data request, exchange and classification) that must be developed with accuracy and certificated. By using or extending fundamental primitives is possible to construct incremental and interchangeable blocks that can be used to perform all the other services requested.
\subsubsection{Portability}
Data4Help service can be reached by third parties from http request, so every browser can be perform request and retrieve users' data.
AutomatedSOS and Track4Run applications are developed as multi platform technology so either iOS or Android devices can run these two apps. (remember that AutomateSOS requires Smartwatch device).


