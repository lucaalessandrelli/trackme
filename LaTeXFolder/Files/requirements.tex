Organize this section according to the rules defined in the project description. 
\subsection{External Interface Requirements}
\subsubsection{User Interfaces}
\subsubsection{Hardware Interfaces}
\subsubsection{Software Interfaces}
\subsubsection{Communication Interfaces}


\subsection{Functional Requirements}
\begin{enumerate}
\item[•]{\Large Data4Help}
	\begin{enumerate}
	\item [G.1] \textbf{Collect users' position and health status.}
		\begin{enumerate}
		\item [D.1] Users' information are collected from partner applications or from the other two TrackMe applications installed on users' devices.
		\item [D.2] All the partner applications require to submit user credentials.
		\item [D.3] The identification (fiscal code, social security number) and the secondary data (attributes) given by the individual during the registration are correct.
		\item [R.1] Retrieve user credentials inserted into partner application as group attributes.
		\item [R.2] Allow individuals to become registered users when policy is approved (first part). Registered users, now, can be tracked in group mode request.  
		\item [R.3] Allow individuals to specify, during registration, if they are also interested to be tracked in single mode request (second part).
		\item [D.4] Devices used to monitor individuals always work and report 			the correct values.	
    	\item [D.5] Partner application always report correct values to Data4Help.
    	\item [R.4] The system has to correctly receive data from partner applications installed on users' device.
    	\end{enumerate}	
    	
    \item [G.2] \textbf{Provide to Third Parties, the users' position and heath status.}
    	\begin{enumerate} 
    	\item [R.5] Allow third parties registration to Data4Help service, where they have to specify all their credentials.
    	\end{enumerate}	
		
		\begin{enumerate} 
		\item [G.2.1] \textbf{Provide data on-demand to non-subscribed third parties.}
		\begin{enumerate} 
		\item [R.6] For each user registered ,the system has to automatically retrieve and store data from partner applications with a resolution of 10 minutes.	
		\item [R.7] The system has to collect inside his database all the useful information that match the request.
		\item [R.8] The system has to send to applicant all the data already collected.
    	\end{enumerate}	
    	
    	\item [G.2.2] \textbf{Provide data in real-time to subscribed third parties.}
		\begin{enumerate}
		\item [D.8] Live acquisition lasts 24 hours to reduce waste of resources.
    	\item [R.9] Allow third parties subscription to interested group in order to receive live data.
    	\item [R.10] When a real time request is performed the system has to collect and store specific users' data with a resolution of 1 minute.
    	\item [R.11] Provide to subscribed third parties data as soon as they are available by the system.
    	\end{enumerate}
    	\end{enumerate}
    
	\item [G.3] \textbf{Allow third parties two different ways to get users' data.}
		\begin{enumerate}     
    	\item [G.3.1] \textbf{Allow third parties to get data of a single person.}
		\begin{enumerate}
		\item [D.6] In order to perform an individual request, third parties has to know the user's fiscal code or security number.
		\item [D.7] Security number and fiscal code are not information given to third parties by Data4Help.
    	\item [R.12] Allow third parties to insert fiscal code of user that want to track.
    	\item [R.13] Deny third parties to receive information about users in  single mode, that have not accepted second part of privacy policy.
    	\item [R.14] Collect all the useful information retrieved by Data4Help that are produced by the interested users 
    	\item [R.15] Send all the collected information to request applicant.
    	\end{enumerate}
    
    	\item [G.3.2] \textbf{Allow third parties to get data of a group of people.}
		\begin{enumerate}
    	\item [R.16] Allow third parties to insert attributes in which they are interested to restrict their field of search.
    	\item [R.17] Deny third parties to receive information if the provided information can hurt users' privacy, for this purpose group request under 1000 users involved are rejected.
    	\item [R.14] Collect all the useful information retrieved by Data4Help that are produced by the interested users 
    	\item [R.15] Send all the collected information to request applicant.
    	\end{enumerate}
    	\end{enumerate}
    	
    \item [G.4] \textbf{Provide data in an anonymous way, to protect users' privacy.}
		\begin{enumerate}
    	\item [R.13] Deny third parties to receive information about users in  single mode, that have not accepted second part of privacy policy.
    	\item [R.17] Deny third parties to receive information if the provided information can hurt users' privacy, for this purpose group request under 1000 users involved are rejected.
    	\end{enumerate}	
			
	\end{enumerate}
	
	
\item[•]{\Large AutomatedSOS}
	
	\begin{enumerate}
	\item [G.5] \textbf{Retrieve user's position and health status.}
		\begin{enumerate}
		\item [R.18] Allow users to be tracked from AutomatedSOS filling up the registration and agreeing to both parts of privacy policy.
		\item [D.4] Devices used to monitor individuals always report correct values.
		\item [D.9] The user always dresses a smartwatch on which AutomatedSOS is installed.    
		\item [R.19] The application has to interact with Smartwatch/Smartphone APIs in order to retrieve location and health status.
		\item [R.20] The application is able to send to Data4Help service all the informations already retrieved in live acquisition.
		\end{enumerate}
		
	\item [G.6] \textbf{Allow health-interested third parties the access to data detected by AutomatedSOS.}
		\begin{enumerate}
		\item [R.21] Allow non-profit organizations to register into AutoatedSOS portal and becoming health third parties.
		\item [R.22] Allow health third parties to receive informations about all the users registered to AutomatedSOS through Live Acquisition performed by Data4Help.
		\end{enumerate}
	
	\item [G.7] \textbf{Monitor user's health parameters.}
		\begin{enumerate}
		\item [R.19] The application has to interact with Smartwatch/Smartphone APIs in order to retrieve location and health status.
		\end{enumerate}
		
	\item [G.8] \textbf{Send an ambulance to users' location whenever certain parameters are below the threshold.}
		\begin{enumerate}
		\item [R.24] The application has to control health status with data retrieved in local to realize immediately if certain parameters are critical.
		\item [R.25] The application has to call an ambulance, if parameters are critical.
		\item [D.10] The ambulance system is always up and ready to receive messages from AutomatedSOS.
		\item [R.26] Supply to hospital user's location and all the useful information to provide efficient first aid.
		\item [D.11] The ambulance successfully reach the location of the individual.
		\end{enumerate}
		
  	\end{enumerate}
  	
  	
\item[•]{\Large Track4Run}
	
	\begin{enumerate}
	\item [G.5] \textbf{Retrieve user's position and health status.}
		\begin{enumerate}
		\item [R.27] Allow users to be tracked from Track4Run filling up the registration and agreeing to both parts of privacy policy.
		\item [D.4] Devices used to monitor individuals always report correct values.
		\item [R.19] The application has to interact with Smartwatch/Smartphone APIs in order to retrieve location and health status.
		\item [R.20] The application is able to send to Data4Help service all the informations already retrieved in live acquisition.
		\end{enumerate}
		
	\item [G.9] \textbf{Allow promoters to manage a run.}
		\begin{enumerate}
		\item [R.21] Allow users to create a run providing all the general information about the competition.
		\item [R.22] Allow users to specify if the race is public or private.
		\item [D.4] Devices used to monitor individuals always report correct values.
		\item [D.13] During a run athletes always dress a smartwatch on which Track4Run is installed.
			
		\item [G.9.1] \textbf{Allow promoters to define a path for the run.}
			\begin{enumerate}
			\item [R.21] Allow promoters to define a path for the race by selecting the routes inside a map.
			\item [D.14] The path defined by the organizer actually exist.
			\end{enumerate}
			
		\item [G.9.2] \textbf{Allow promoters to invite athletes to the run.}
			\begin{enumerate}
			\item [R.21] Allow promoters to invite athlete to be runner of their private race.
			\item [R.22] Allow promoters to specify maximum number of athletes that can take part to their public race.
			\end{enumerate}
	\end{enumerate}
	
	\item [G.10] \textbf{Allow athletes to enroll on a specific run.}
		\begin{enumerate}
		\item [R.23] Allow users to see all the public races and private races in which he is invited.
		\item [R.24] Allow user to select a race and add him to the athletes involved.
		\item [D.16] If an athlete enroll to a run then he also participates to the run.
		\end{enumerate}
	
	\item [G.11] \textbf{Allow spectators to watch in real time the position of every athletes in a specific run.}
		\begin{enumerate}
		\item [D.17] All athletes have their tracking devices with them and the application enabled for the entire duration of the run.	
		\item [R.25] Allow user to select a race to be viewed.
		\item [R.26] The application must be able to request to Data4Help the positions of all the other athletes involved.
		\item [R.27] The application must be able to receive and display the positions of all the other athletes involved.
		\item [D.18] Athletes never go out of the defined path.
		\end{enumerate}
	\end{enumerate}

\end{enumerate}




\subsection{Performance Requirements}

\subsection{Design Constraints}
\subsubsection{Standards compliance}
\subsubsection{Hardware limitations}
\subsubsection{Any other constraint}

\subsection{Software System Attributes}
\subsubsection{Reliability}
\subsubsection{Availability}
\subsubsection{Security}
\subsubsection{Maintainability}
\subsubsection{Portability}
